
There are a few options available, but it depends on how much you'd like to pay.
On the more budget end, the BrewPI is a DIY kit that you could put together. You
have to buy all the parts seperately, but they are all available off of one
website. The BrewPI Spark is capable of controlling the temperature of a
"hacked" fridge, the temperature of the heater inside of the fridge, and it
monitors the temperature of the beer itself. The BrewPI Spark then communicates
with another Raspberry PI that runs a web server that can be viewed on a
computer or phone. It does not have mash temperature control or a way to control
pumps or valves. The system has commercially available parts, but there is a
very large do-it-yourself component to this project. All the parts can be
purchased for around nine hundred dollars. \cite{Jacobs_2017} 

The Pico Pro is an "all in one" brewing machine. You don't need to purchase
pieces of a kit and put it together yourself. The brewing capacity is 1.75
gallons. It handles the heating of liquids
and transfers the contents of the machine into a small keg via a pump. It
doesn't offer any special capabilities other than being able to scan proprietary
PicoBrew recipes. Setup is very minimal, there is not a lot of components that
need to be put together. The product is commercially available for five hundred
and fifty dollars. \cite{Pico_2021}

The Grainfather G30 uses a single chamber for all of it's brewing except when it
chills the wort in the wort chiller. The brewing capacity of the Grainfather is
a little over 6 gallons. The grain father is all electric and is
intended to be used mainly indoors. \cite{AllGrain_2021} It has a phone application that is used to
control heat and power. It has a pump for transferring liquids between the main
vessel and the wort chiller. There is minimal set up and the product can be
purchased for one thousand dollars. \cite{Grainfather_2019} 
        
The Brewie is one of the more expensive options. It has two very large chambers
for brewing. It is capable of mashing, sparging, hopping, cooling, draining, and
fermentation. It has built in weight sensors, pressure sensors, self cleaning
capabilities, and a built in cooler for wort chilling. It has a brewing capacity
of up to 5.28 gallons. Set up is minimal and the product can be purchased for
one thousand five hundred dollars. \cite{MoreBeer_2020}

The closest competitor to our product is the BrewPI. The big disadvantage
that it has is that you need to buy each individual part from BrewPI, then put it
together yourself. There could arise an issue with sourcing so many different parts from
a single small scale business. We want to offer a similar product, but with
significantly less reliance on doing it yourself. Our team is positive we can
produce a product that costs well under the full price of the BrewPI. 
