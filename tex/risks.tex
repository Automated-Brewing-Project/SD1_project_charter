This section should contain a list of at least 5 of the most critical risks related to your project. Additionally, the probability of occurrence, size of loss, and risk exposure should be listed. For size of loss, express units as the number of days by which the project schedule would be delayed. For risk exposure, multiply the size of loss by the probability of occurrence to obtain the exposure in days. For example:

The following high-level risk census contains identified project risks with the highest exposure. Mitigation strategies will be discussed in future planning sessions.

\begin{table}[h]
\resizebox{\textwidth}{!}{
\begin{tabular}{|l|l|l|l|}
\hline
 \textbf{Risk description} & \textbf{Probability} & \textbf{Loss (days)} & \textbf{Exposure (days)} \\ \hline
 COVID 19 Exposure before assembly or in person testing   & 0.20 & 14 & 2.8 \\ \hline
 Leaking parts damage electrical equipment  & 0.30 & 6 & 1.8 \\ \hline
 Shipping delays from COVID 19 for certain hardware components  & 0.20 & 10 & 2.0 \\ \hline
 Part incompatibility  & 0.30 & 3 & 0.9 \\ \hline
 Unforeseen requirements for hardware & 0.10 & 3 & 0.3 \\ \hline
\end{tabular}}
\caption{Overview of highest exposure project risks} 
\end{table}